%! Author = felix
%! Date = 09.05.25

\documentclass[11pt]{article}

\usepackage{amsmath}
\usepackage[utf8]{inputenc}
\usepackage{amssymb}

\title{Command History in Collaborative Systems}

\begin{document}
    \section{Theory}

    I am currently working on a Logic-Gate Simulation Game called BuildIT.
    Players can move components on the circuit board around, create new ones, or delete old ones.
    They can rotate components, change settings, and a lot of other actions.
    All of these actions that come from the View (e.g. MoveComponents, RotateComponents) must be
    reversible.
    Other systems use the Command Pattern for that.
    This means that for every action an inverted action must exist or can be derived.
    This is quite a bit of work to manage reliable and also causes other problems.

    \paragraph{}
    For my game I am using an ECS, an Entity Component System.
    The entire state of the application is stored in the ECS.
    That is why I decided that while an action is executed and is modifying the state,
    I just record all changes during that execution and attach them to the action.
    Then I can freely undo / redo these changes without having to implement a single inverted action.
    I think about the ECS more of a Database.
    And if the important state is represented in the Database, why not use the
    changes in the state itself for the history.

    \paragraph{}
    In most of the text editors a Linear undo model is implemented.
    There exists an undo and a redo stack.
    When undoing an action, pop it from the undo stack and push it to the redo stack.
    When redoing do the opposite.
    This works quite well, but for collaborative Systems a Linear undo model does not work anymore.
    The user expects to undo and redo his own actions, not all actions pushed to a global history.
    And with that a whole pile of problems emerges.
    Suddenly, we must be able to undo an action, even though another user maybe caused actions after it.
    The history must become non-linear.
    Now there have to be some rules.

    \paragraph{}
    Regarding BuildIT, if Player 1 moves one Component (which causes an action) and
    Player 2 moves another Component (which causes another action), Player 1 should
    be able to freely undo his action even though another action was issued after it, because
    Player 2's action does not interfere with Player 1's action.
    On the other hand, if Player 1 moves one Component and Player 2 deletes that exact same Component,
    Player 1 should and will not be able to undo unless Player 2 undoes his action,
    so the Component is created again.
    So rule No. 1 is that undoing or redoing an action may only happen if all following actions that modified
    components referenced by that action were reverted before.

    \paragraph{}
    Our ECS makes things a lot easier.
    Because the ECS consists of a lot of entities, if two actions $A$ and $B$ do not share any modified entities,
    we can consider them as independent.
    Because I like math and mathematical notations, I will define the following:
    Each action $A$ is an ordered set of changes $c$ with $c \in A$.
    If two actions $A$ and $B$ are independent, we write $A \cap B = \varnothing $.
    That does not really help us, but it looks cool!

    \paragraph{}
    Before I describe the implementation, note that while the history across all users is non-linear,
    my implementation is linear per-player, which is not mandatory, but I decided to do so.
    I still do not stick to the typical Undo/Redo stack approach, but to an approach called History undo
    which emacs also uses.
    Emacs' History undo solves the issue that when you go back in time by undoing some actions and issue
    a new action, all the undone actions get lost.
    It works by using a pointer, which points to the next action to be undone.
    If a user undoes an action, the inverted action is appended to the linear history (which exists in my case per-player)
    and the cursor moves one to the left (back in history).
    If a user redoes an action, the previously undone action is executed again and the inverted action deleted from the history.
    If a user undoes an action $A$, so the History becomes $A \rightarrow \overline{A}$ and
    should then decide to issue another action $B$, the history would become $A \rightarrow \overline{A} \rightarrow B$, preserving
    the undone action!

    \section{Implementation}
    Because we are recording possibly thousands of changes per action, we want the history to be as memory-efficient as possible.
    At the same time, we want it to be somewhat performant.
    This is a tradeoff you often have to make, and the future will show if my approach is `the golden mean'.

    For each player I manage the actions the player issued and a pointer, which points to the next action to be
    undone as described in the last section.
    Also, each action gets a unique id (32-bit unsigned integer in my case).
    For each entity I store the last action-id, which modified that entity in the LMAC (Last Modified Action Cache).
    Each change in an action stores the action-id the entity had before the change and after the change in the LMAC.
    When inverting an action, I reverse the changes and switch old-action-id and new-action-id in each change.
    When executing an action and thus each change in the action, the new action-id in the change gets stored in the LMAC.
    Because, when undoing an action, the reversed action gets executed, the old action-id gets stored in the LMAC automatically.
    To follow rule No. 1 as defined in the last section, we can only undo or redo an action if
    the action-id stored in the LMAC for all entities referenced by that action equals the old action-id referenced in the change.

    \subsection{Example}
    Player 1 Moves Component on the circuit board.
    This starts a new action (changes to the ECS get `recorded' now).
    The started action gets, for example, action-id 1.
    The movement causes a change to be created and added to the action with an old action-id of 0 (previous action)
    and the new action-id 1 (the current action-id of the action being recorded).
    Finishing the action will iterate through all recorded changes and set the
    action-id in the LMAC for all modified entities to the action-id of the currently recorded action (1).
    Player 2 now deletes the Component on the circuit board.
    This starts a new action (changes to the ECS get `recorded' now) again.
    The started action gets action-id 2.
    The deletion causes a change to be created and added to the action with an old action-id of 1 (previous action)
    and the new action-id 2 (the current action-id of the action being recorded).
    We again finish the action, which means the LMAC now stores action-id 2 for the entity.
    Now Player 1 wants to undo the movement.
    This will show an error to the user, as undoing Player 1's action requires an action-id of 1 stored in the LMAC, but
    currently action-id 2 is stored.
    The only way to resolve the conflict would be that Player 2 undoes its action, which would cause action-id 1 to be written
    to the LMAC for the entity.
    Then Player 1 could undo his action freely.
    Note that when Player 1 now undoes his action action-id 0 would get stored in the LMAC.
    This means that Player 2 cannot redo his action anymore (because it would require an action-id of 1), unless Player 1
    redoes his action first, which would cause action-id 1 to be stored in the LMAC, which makes complete sense!
\end{document}