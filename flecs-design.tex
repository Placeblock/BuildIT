\documentclass{article}

\usepackage{listings}
\usepackage[utf8]{inputenc}

\title{Gedanken zur Verwaltung des Zustandes einer digitalen Leiterplatte in ECS}

\begin{document}
    \section{ECS}
    Ein Entity Component System (kurz ECS) ist ein Software-Architekturmuster zur
    Representation von Entitäten in Spielen.
    Dabei wird der Ansatz Composition over Inheritance verfolgt.
    Das bedeutet, jedes Entity ist nicht durch vererbte Klassen definiert,
    sondern durch dem Entity zugeordnete Komponenten.
    Oft wird auch die Bezeichnung `Data-Driven' verwendet, da ein Entity inkl. Komponenten
    keinerlei Funktionalität besitzt, sondern allein den Zustand dieses Entities beschreibt.

    \subsection{Begriffe}
    Objekte im Sinne des ECS sind Entitäten.
    Komponenten beschreiben diese Entitäten.
    Im Sinne meines Logik-Gatter-Simulations-Spiels beschreiben Entitäten die einzelnen Komponenten
    auf der Leiterplatte.
    Es gibt auch einige ECS Komponenten.
    So z. B. die `Position' Komponente, welche die Position der Komponente auf der Leiterplatte speichert.
    Die `Rotation' Komponente, welche die Rotation speichert.
    Das ermöglicht es, die verschiedensten Leiterplatten Komponenten zu entwickeln, ohne komplexere hierarchische
    Vererbungsstrukturen zu benötigen.
    Wenn eine Komponente nicht gedreht werden können soll, fügt man einfach die `Rotation' Komponente nicht hinzu.
    ECS Systeme wirken über die ECS Entitäten hinweg, fügen Komponenten zur Laufzeit hinzu, entfernen diese oder modifizieren sie.
    Es gibt keine rotate Methode im Entity selbst, welche es dreht, wie man es aus klassischen Objekt Orientierten Programmen
    kennt, sondern es würde ein externes Rotate-System geben, welches von außen die Rotations-Komponenten der Entitäten modifiziert.
    Hier zeigt sich auch wieder, dass die Entitäten die reinen Daten speichern.
    Dieses System funktioniert durch das Abfragen von Entitäten mit bestimmten Eigenschaften.
    Durch Queries kann sehr leicht über
    alle Entitäten iteriert werden, die bestimmte Komponenten besitzen.
    Bei Space Invaders könnte jedes Bullet durch ein Entity mit der `Position' Komponente versehen sein, die die aktuelle Position enthält.
    Außerdem könnte noch eine leere Komponente `Bullet' hinzugefügt werden, um den Typ dieses Entities zu kennzeichnen.
    Jetzt sollen alle Bullets im Spiel kontinuierlich nach oben bewegt werden.
    Dafür schreiben wir ein System, welches über alle Entitäten
    mit den Komponenten Position und Bullet versehen ist, und aktualisieren die Positions-Komponente.

    In dem ECS Flecs sieht das wie folgt aus:
    \begin{lstlisting}[language=C++]
struct Position {
    int x;
    int y;
};

flecs::world world;
flecs::entity entity = world.entity();
entity.set<Position>({0, 0});

world.system<Bullet, Position>().each([](Bullet& bullet, Position& pos) {
    pos.y -= 10;
    // Invoked for each entity with Position and Bullet Components
});
    \end{lstlisting}

    Diese Abfragen sind ggf. auch gecached, das heist mit dem Erstellen eines Systems und somit einer Query nach Entitäten
    mit bestimmten Eigenschaften wird eine effiziente Datenstruktur im ECS verwaltet, welche die Iteration der Entitäten
    mit diesen Eigenschaften extrem schnell macht.

    Viele ECS und so auch Flecs sind eigentlich auf sogenannte Tick-Basierte Spiele ausgelegt.
    In einer Welt existieren sehr, sehr, sehr viele Entitäten und in jedem Tick soll mit diesen Entitäten etwas geschehen.
    Dafür stellt Flecs die sogenannten Pipelines bereit, mit welchen man mehrere Stufen definieren und Systeme diesen Stufen zuordnen kann.
    Dann ruft man jeden Tick diese Pipeline auf und die Systeme werden in der richtigen Reihenfolge ausgeführt.
    Sogar für das wiederholte Ausführen dieser Pipelines bzw. einzelner Systeme stellt Flecs funktionen bereit.

    Für das Space-Invader Beispiel könnte die Pipeline des Spiels, die wie die Main-Loop in klassisch entwickelten Spielen
    angesehen werden kann, das oben beschriebene System beinhalten, welches Bullets nach oben bewegt.
    Ein weiteres System zur Kollisions-Erkennung usw. wären eine Leichtigkeit.

    Mit etlichen weiteren Features wie zum Beispiel einfachen Beziehungen zwischen Entitäten, Entity Templates,
    Observern, um auf Änderungen in einer ECS Welt zu reagieren, komplexe Queries für Entitäten, usw. können somit
    große, komplexe, Entity-Basierte Spiele mit einem Data-Driven Ansatz strukturiert und übersichtlich umgesetzt werden.
    
    \section{Anwendung von ECS Funktionen in BuildIT}
    Die Anwendung eines ECS ist aus vielen Gründen vorteilhaft:
    \begin{itemize}
        \item Durch Plug-Ins sollen weitere Komponenten für die Leiterplatte hinzugefügt werden können. Diese sollen möglichst
              flexibel programmierbar sein und können im ECS einfach durch eine Ansammlung von verschiedenen Komponenten dargestellt werden.
              Zusätzlich kann der Plug-In Entwickler einfach ECS Systeme entwickeln, welche ggf. Komponenten des Plugins modifizieren.
        \item Wenn Leiterplatten-Komponenten über das Netzwerk synchronisiert werden müssen, ist eine eindeutige über das Netzwerk hinweg
              stabile Entity-ID notwendig. Diese könnte in Form eines `NetworkID' Komponenten realisierbar, dass aber nur dem Entity hinzu-
              gefügt werden muss, wenn über das Netzwerk gespielt wird.
        \item Wenn über das Netzwerk gespielt wird, muss jeder Client eine Representation der ECS-World besitzen, um den Zustand auf dem
              Bildschirm darzustellen. Dafür sind ggf. eigene Komponenten von Nutzen, die aber nur benötigt werden, wenn eine Entität
              auch dargestellt werden soll. Auf dem Server könnte man so einfach auf diese Komponenten verzichten
    \end{itemize}
    Es zeigt sich also, dass ein ECS für hoch-flexible und erweiterbare Spiele eine gute Lösung sind, um diesen Grad der Flexibilität trotzdem
    bei guter Code-Qualität zu gewährleisten.

    Wie bereits am Anfang erwähnt stelle ich jede Leiterplatte-Komponente durch eine ECS-Entität dar, inkl. Positions Komponente und ggf.
    Rotations-Komponente.
    Ein Entity kann ggf. mehrere Pin-Komponenten enthalten, welche die Input- und Output-Pins dieses Entities beschreiben.
    Ein `Joint' - der Endpunkt bzw. der Knoten zwischen mehreren Kabeln - ist ebenfalls eine ECS-Entität, mit einer zusätzlichen `Joint'-Komponente,
    welche eine Referenz auf eine Pin-Komponente enthält, falls dieser Joint auf einem Pin liegt.

    \subsection{Operieren auf dem ECS}
    Dieser Zustand der Entitäten muss nun verändert werden.
    Wenn ein Spieler eine Komponente bewegt, soll die zugehörige ECS Positions-Komponente
    angepasst werden.
    Da hängt jetzt allerdings viel dran.
    Was ist, wenn ein Spieler einen Joint von einem Pin einer Komponente wegbewegt?
    Dann muss die zugehörige Verbindung in der Simulation getrennt werden und die Pin-Referenz in der Joint-Komponente entfernt werden.
    Was passiert, wenn ein Joint als solches gelöscht wird?
    Ich benötige im Programm Funktionalität, um zusammengehörige Joints und Kabel in sogenannten
    Netzwerken zusammenzufassen.
    Sollte ein Joint gelöscht werden, könnte es passieren, dass ein Netzwerk in zwei neue Netwerke aufgeteilt werden muss.
    Ob ein Netzwerk durch eine weitere Entität dargestellt wird, ist noch unklar, könnte aber einige Vorteile bieten.
    Z. b. würde es ermöglichen,
    Joint und Wire Entitäten als Kinder dieses Netzwerks Entität zu setzen, was gewisse Abfragen leichter macht.
    Diese Funktionalität bringt Flecs als
    ECS mit.
    Das würde zum Beispiel zur Folge haben, dass, wenn die Netzwerk-Entität gelöscht wird, automatisch alle Kinder, also Wire und Joint Entitäten,
    automatisch mitgelöscht werden.

    Änderungen an diesem System müssen also klar definiert, strukturiert und ohne große Seiteneffekte geschehen, um Resultate wie in Version 1 des
    Programms zu vermeiden ;).
    Bereits in Version 1 wurden sogenannte Events eingeführt.
    Alle Interaktionen, welche vom Spieler erfolgen, lösen klar definierte Events aus:
    \begin{itemize}
        \item Move - Enthält eine Liste von Entitäten, welche bewegt wurden und zusätzlich einen Delta-Wert
        \item Create - Erstellt eine Komponente
        \item Delete - Löscht eine oder mehrere Komponenten
        \item CreateMacro - Erstellt eine Makro-Komponente
        \item \dots
    \end{itemize}
    Jedes Event muss jetzt eine bestimmte Funktionalität implementieren.
    Indem man für jedes Event die `umgekehrte' Funktionalität zusätzlich implementiert,
    ermöglicht man außerdem sehr einfach das Bereitstellen einer History mit Undo- und Redo-Funktionen.

    Leiterplatten-Komponenten können allerdings nicht einfach verschoben werden, sondern die Verbindungen zu Netzwerken,
    also Kabeln und somit anderen Komponenten, müssen beachtet werden.
    Wenn ich zwei Komponenten auswähle und sie so verschiebe,
    dass Komponente 1 nach dem Verschieben an der gleichen Stelle ist wie Komponente 2, zu welcher ein Kabel geführt hat,
    dann müssen alte Verbindungen zu Komponente 2 korrekt getrennt werden, und Verbindungen zu Komponente 1 nach dem Verschieben
    hergestellt werden.
    Dafür wird das Bewegen von Komponenten in drei Schritte aufgeteilt:
    \begin{enumerate}
        \item Trennen bestehender Verbindungen.
        \item Verschieben der Komponenten.
        \item Aufbauen entstandener Verbindungen.
    \end{enumerate}
    Diese drei Schritte müssen beim Eingang eines Move-Events also ausgeführt werden. Dafür können ECS-Queries geschickt
    genutzt werden, um die Pins aller zu verschiebenden Entitäten mit Pin ECS-Komponenten ggf. von Joints zu trennen und nach
    dem Bewegen wiederherzustellen.
    Auserdem muss sichergestellt sein, dass diese Events in der Reihenfolge abgearbeitet werden, wie sie `ankommen'.

    \subsection{Observer Versuch}
    In Version 1 habe ich viel versucht über das Observer-Pattern umzusetzen, was früher oder später zu einem reinen Event-Chaos
    geführt hat.
    Das Problem war allerdings vor allem, dass Leiterplatten-Komponenten gegenseitig auf Nachrichten von anderen Komponenten
    gehört haben.
    Der Gedanke war ein Observer, welcher Positions-Änderungen von Komponenten erfährt, und falls Joints von Pins (oder andersherum)
    getrennt wurden oder entstanden sind, die Simulation entsprechend anzupassen.
    Wenn man kurz darüber nachdenkt, führt das allerdings schnell zu dem oben beschriebenen Problem, dass ich erst alle bestehenden Verbindungen
    von zu bewegenden Komponenten trennen, dann alle bewegen und anschließend die Verbindungen wiederherstellen muss.
    Durch den Observer würden beim Bewegen einer einzelnen Komponente sofort alle Verbindungen aktualisiert werden.
    Mit reinen Observern ist das allerdings nicht möglich und so ist in Version 1 die Logik zum Synchronisieren der Verbindungen
    bei Verschieben von Komponenten in die Logik zum Verschieben selbst gewandert.
    Vom Aktualisieren der Netzwerke beim Löschen oder Erstellen von Kabeln habe ich bisher gar nicht geschrieben.

    \subsection{Systeme}
    Nun bin ich also zu dem Entschluss gekommen, dass Observer nicht die beste Wahl sind.
    Sie machen es zusätzlich schwer den Programm-Fluss
    nachzuvollziehen und da hatte ich ja bereits festgestellt, dass das hier besonders wichtig ist.
    Durch die Nutzung des ECS können wir geschickt Nutzen von Systemen machen, um unser Vorhaben in die Tat umzusetzen.
    Für das Move-Event schreiben wir drei Systeme, welche die drei Punkte (Trennen, Verschieben, Verbinden) jeweils ausführen.
    Sollte nun ein Move-Event angestoßen werden, führen wir diese drei Systeme der Reihe nach aus.
    Stellt sich die Frage, warum man dafür Systeme verwenden sollte.
    Eine trivialere Möglichkeit wäre es, einfach dreimal über die zu bewegenden Entitäten zu iterieren.
    Das wäre hier auch ggf. Sinnvoller, allerdings haben Systeme die Eigenschaft mit Queries zu arbeiten.
    Es gibt keine Möglichkeit, ein System auf alle zu bewegenden Entitäten auszuführen.
    Systeme operieren auf Entitäten, welche bestimmte Eigenschaften erfüllen.
    So müsste man zu bewegenden Entitäten eine `ToMove' Komponente hinzufügen,
    welche den Delta-Wert, um den verschoben werden soll, enthält.
    Die Systeme können dann über alle Entitäten mit der `ToMove' Komponente iterieren und ihre Funktionalität implementieren.
    Das klingt jetzt alles sehr komplex und unübersichtlich, also eigentlich genau das Gegenteil von dem, was wir erreichen wollen.
    Da wären drei for-Schleifen für die drei Schritte doch einfacher, möchte man meinen.

    \subsection{Pipelines und Tick-Basierte Spiele}
    Jetzt bin ich erneut über die Pipelines gestoßen.
    Ich könnte für die gesamte Leiterplatte eine Pipeline aus Systemen erstellen, welche in definierter Reihenfolge ablaufen.
    Das erste System verarbeitet seit dem letzten Pipeline-Durchlauf eingegangene Events.
    Das Move-Event setzt z. B. die `ToMove' Komponente.
    Das Delete-Event setzt z. B. die `ToDelete' Komponente.
    Danach kommen die drei Systeme, welche sich um das Verschieben und Aktualisieren der Verbindungen kümmern.
    Anschließend könnte ein System die bereits beschriebenen Netzwerke aktualisieren.
    Dann löscht ein System die zu löschenden Entitäten.
    Ein System könnte z. B. zum Schluss den Zustand des Spiels abspeichern.
    So ist sichergestellt, dass das Abspeichern nur zu definierten Zeitpunkten geschieht, was sonst über z. B. Locks
    oder Ähnliches sichergestellt werden müsste.
    Also eine feste Pipeline aus System, die sich um alles kümmert, was irgendwie geschehen kann.
    Das würde z. B. auch den großen Vorteil bieten, dass auch andere Events oder Plug-Ins falls nötig einfach eine
    `ToMove' Komponente setzen könnte.
    Die Pipeline würde sicherstellen, dass die Verarbeitung dieser Komponente zum richtigen Zeitpunkt geschieht.
    Da im alten System die Bewegung von Komponenten im Move-Event selbst geschehen ist,  waren solche Dinge unmöglich.

    Der Pipeline ansatz hat allerdings noch ein kleines Problem.
    Bisher habe ich beschrieben, dass in der ersten `Stage' alle seit dem letzten Pipeline-Durchlauf eingegangenen Events verarbeitet werden.
    Das verletzt allerdings meine Regel, dass Events in der richtigen Reihenfolge bearbeitet werden müssen, da das eigentliche Verschieben
    nun nicht mehr direkt bei der Verarbeitung des Events geschieht, sondern zu einem späteren Zeitpunkt.
    Wenn also in einem Pipeline-Durchlauf eine ungünstige Konstellation an Events verarbeitet wird, könnte es zu Inkonsistenzen im Zustand
    des ECS führen.
    Die einzige Lösung wäre es, diese vordefinierte Pipeline einfach bei Eingang eines Events aufzurufen, sodass pro Pipeline-Durchgang
    nur ein Event verarbeitet wird.
    Somit wäre der Ablauf für Änderungen am ECS durch die Pipeline klar definiert, durch Systeme sinnvoll
    gekapselt und getrennt von den eigentlichen Daten, die Reihenfolge der Verarbeitung der Events allerdings trotzdem gewährleistet.
    Das würde natürlich bedeuten, dass für ein Event Systeme ausgeführt werden, die für dieses Event ggf. gar keine Rolle spielen.
    Jetzt stellt sich natürlich die Frage, ob das nicht sogar von Vorteil sein kann.
    Denn ein System, welches nichts zu tun hat, benötigt sowieso kaum Rechenzeit.
    Außerdem würde diese feste Pipeline dazuführen, dass, selbst wenn ein Event mal eine Leiterplatten-Komponente durch
    eine `ToMove` ECS-Komponente verschieben will, sichergestellt ist, dass alles ordnungsgemäß und eben in einer klaren
    und ersichtlichen Reihenfolge verarbeitet wird.

\end{document}