%! Author = felix
%! Date = 02.05.25

% Preamble
\documentclass[11pt]{article}

% Packages
\usepackage{amsmath}
\usepackage[ngerman]{babel}
\usepackage[T1]{fontenc}

\title{Event Verarbeitung und Änderung am ECS}

% Document
\begin{document}

    \section{Event Verarbeitung und Änderung am ECS}
    Der Memento Ansatz zur Änderungs-Verwaltung beschreibt den Ansatz, welche unmittelbare Änderungen
    am ECS speichert, um diese Änderungen später einfach wieder rückgängig zu machen.
    Das Command Pattern speichert nicht die Änderungen am Zustand selbst, sondern vordefinierte Events.
    Das setzt voraus, dass für jedes Event $A$ auch ein Inverses Event $\overline{A}$ existiert.
    Für BuildIT verfolge ich das Command-Pattern aus folgenden Gründen:
    \begin{enumerate}
        \item Die gesamten Änderungen des ECS zu speichern, würde erheblich mehr Speicher benötigen, als nur die Event Daten.
        \item Die Änderungen am ECS auf eine abstrakte Art und Weise zu speichern, würde Multiplayer-Undo/Redo Systeme nicht
              auf eine Art und Weise ermöglichen, wie ich es mir vorstelle.
    \end{enumerate}
    Alle Änderungen am ECS müssen demnach durch klar definierte Events erfolgen.
    Verschiebt ein Spieler eine Komponente, muss diese Aktion ein Event auslösen.

    Eine weitere Funktionalität, welche das ECS zuverlässig bereitstellen muss,
    ist das Aktualisieren von Netzwerken,
    das Verbinden und Trennen von Komponenten mit Netzwerken
    und ggf. noch viele andere sogenannte Side-Effects.
    Dabei ist zu erwähnen, dass diese zusätzlichen Funktionalitäten
    nicht vom Spieler selbst, bzw. vom View erzeugt werden.
    Das View sollte lediglich die eigentlich vom Spieler ausgeführte
    Aktion als Event erzeugen, z. B. das Bewegen von Komponenten.
    Stellt der Controller dann fest, dass basierend auf diesem Event
    noch andere Aktionen ausgeführt werden
    sollten, muss sichergestellt werden, dass diese Aktionen trotzdem
    in der Event-History landen, um diese zusätzlichen
    Events bei Undo/Redo Schritte auch zu berücksichtigen.

    \paragraph{}
    Diese sogenannten Side-Effects können erst auftreten, nachdem das ursprüngliche Event ausgeführt wurde, da diese aus
    State-Änderungen hervorgehen.
    Die Side-Effects sollen deshalb durch State-Änderung hervorgerufen werden,
    da sie dann unabhängig vom ursprünglichen Event sind.
    Egal, welches Event dazu geführt hat, dass zwei Netzwerke verbunden werden müssen,
    es wird durch den Side-Effect geschehen.
    Das ist für die Zuverlässigkeit und Sicherheit des Systems wichtig.

    \paragraph{}
    Der Nachteil, der aus dieser reaktiven Funktionsweise hervorgeht, ist,
    dass vor dem Ausführen eines Events nicht bekannt ist,
    welche Seiten-Effekte dieses Event ggf. nach sich zieht.
    Das wäre nur möglich, wenn eine Kopie des Zustands erzeugt werden würde und
    das Event inkl. Side-Effect zunächst in dieser Kopie "getestet" werden kann,
    was allerdings nicht praktikabel ist.
    Diese Trennung der Side-Effects von den eigentlichen Events und das
    Sicherstellen des Auslösens von Side-Effects unabhängig von den Events hat
    diesen Nachteil der Unvorhersehbarkeit als direkte Folge.
    Es ist logisch nicht möglich, unabhängig vom Event die folgenden
    Seiten-Effekte zu bestimmen, ohne die Funktionsweise des Events zu kennen,
    oder das Event auszuführen und zu beobachten, was passiert.

    \paragraph{}
    Die einzige Möglichkeit ist also, die Side-Effects reaktiv durch Änderungen
    am ECS zu erzeugen und dem ursprünglichen Event zuzuordnen.
    Sollte ein Side-Effect fehlschlagen, z. B. dadurch, dass ein Side-Effect
    feststellt, dass ein Netzwerk bereits einen Eingang besitzt,
    dann muss dieser Side-Effect fehlschlagen, was zur Folge haben muss,
    dass alle bereits ausgeführten Seiten-Effekte inklusive des ursprünglich vom
    Spieler erzeugten Events rückgängig gemacht werden müssen.

    \paragraph{}
    Wie bereits erwähnt ist dieser reaktive Ansatz notwendig,
    um das Ausführen dieser Side-Effects Event-Unabhängig zu gewährleisten.
    Soll zusätzlich zu dieser Funktionsweise vor dem Ausführen eines Events geprüft werden,
    ob das Ausführen des Events möglich ist,
    so muss an der Stelle Event-spezifische Logik eingebaut werden.
    Die Event-spezifische Logik kann dann auch benutzt werden,
    um dem Spieler schon vor dem
    Bewegen mitzuteilen, dass diese Aktion aus diesem und jenem Grund nicht möglich ist.
    So kann außerdem noch die Last auf dem Server verringert werden,
    da der Client das Event schon gar nicht erst schickt,
    wobei die Event-Verarbeitung auf dem Server nicht viel Last erzeugt.

    \subsection{Modifizieren von Daten}
    Wie bereits im letzten Abschnitt gezeigt, ist ein purer Command-Pattern ansatz komplex und birgt viele Risiken.
    Ein ECS ist ein System zum Verwalten von Daten.
    Man kann es eigentlich als In-Memory Datenbank bezeichnen.
    Wieso sollte man das Verwalten dieser Daten nicht auch darauf auslegen, dass ein ECS eigentlich nur "Daten" sind?
    Wenn jeder Aspekt meines Spiels im ECS gespeichert ist, kann das Undo-/Redosystem basierend auf dem Memento-Pattern
    einfach die Änderungen am ECS als zeitlichen Verlauf speichern und so ein flexibles Undo-/Redosystem ermöglichen.
    Wenn ein Spieler ein Event zum Server schickt, wird einfach eine sogenannte Transaktion gestartet.
    Daraufhin werden im ECS alle Änderungen inkl. Side-Effects durchgeführt.
    Alle Änderungen am ECS werden aufgezeichnet.
    Sollte ein Side-Effect fehlschlagen, werden alle Änderungen der Transaktion rückgängig gemacht.
    Wenn eine Transaktion erfolgreich war,
    ist sie als Gruppierung von Änderungen am ECS in der History verfügbar.

    \paragraph{}
    Ein kluger Aufbau der History ermöglicht uns einige coole Dinge.
    Wenn eine Komponente während einer Transaktion zweimal
    verändert wird, kann die erste Änderung vergessen werden!
    Jede Transaktion bekommt einen Tag oder Version.
    Die History wird jetzt für jedes Entity verwaltet.
    Wenn eine Komponente eines Entities verändert wurde, dann wird der neue
    Zustand der Komponente mit dem Tag der Transaktion gespeichert.
    Wichtig ist, dass wir den alten Zustand nicht speichern.
    Wenn eine Transaktion jetzt rückgängig gemacht werden soll, können wir einfach für alle geänderten Komponenten den Zustand
    der vorherigen Transaktion herstellen.
    Wenn keine Transaktion vorher existiert, bedeutet das, dass die Komponente zuvor nicht
    existiert hat, und wir können diese löschen.
    Das ermöglicht es auch, dass, wenn wir mehrere Transaktionen rückgängig machen wollen, wir nicht alle Transaktionen einzeln
    rückgängig machen müssen.
    Wir müssen nur alle Entitäten sammeln, die sich in den Transaktionen verändert haben und dann für jede
    Komponente den Zustand der vorherigen Transaktion herstellen (falls vorhanden, sonst löschen).

    Da die Undo-/Redo History jetzt sehr Entitäten-Nah arbeitet, kann multiplayer History Support leicht eingebaut werden.
    Für multiplayer History Support ist Selective-Undo nötig, das meint die Fähigkeit, beliebige Transaktionen rückgängig zu machen.
    Die Undo-/Redo History erhält jetzt also noch ein Ownership Feature.
    Das bedeutet einfach, dass wir uns in einer Transaktion
    speichern, welcher Nutzer diese Transaktion verursacht hat.
    Wenn ein Nutzer jetzt seine letzte Transaktion rückgängig machen möchte,
    können wir einfach prüfen, ob alle Entitäten, die in dieser Transaktion bearbeitet wurden, auch in der letzten Transaktion immernoch
    diesem Spieler `gehören'.
    Wenn das der Fall ist, kann die Transaktion einfach rückgängig gemacht werden.
    Da auch Netzwerk-Merges, Pin Updates und alle anderen Side-Effects durch Entitäten-Änderungen in der History mit Ownership gehandhabt
    werden, funktioniert das auf diese abstrakte Art und Weise.

    Für Multiplayer Support hat das den großen Vorteil, dass kein zusätzlicher Support für ECS-Synchronisation mehr eingebaut werden muss.
    Die Transaktion wird einfach serialisiert dem Spieler geschickt und diese wird dann einfach auf das lokale ECS des Spielers angewendet.
	Außerdem ermöglicht es Client-Side Prediction.
    Die Transaktionen können auch auf dem Client einfach simuliert werden.
    Sollte auf dem Server etwas in anderer Reihenfolge geschehen sein durch de-synchronisation aufgrund von anderen Spielern,
    dann wird einfach die Transaktion
	zurückgerollt und vom Server in richtiger Reihenfolge synchronisiert.

	Ältere Transaktionen könnten serialisiert und komprimiert gespeichert werden.

    \subsection{IDs überall}
    Um eine stabile History bereitstellen zu können, müssen IDs von Entitäten über die gesamte Laufzeit des Prozesses
    eindeutig und einzigartig sein.
    Mit Prozess ist hier vom Öffnen des Games bis zum Schließen gemeint.
    Wenn die History über das Programm hinaus persistent gespeichert werden soll,
    müssten IDs von Entitäten allerdings über die gesamte Laufzeit des Spiels eindeutig und stabil sein.
    Entitäten im ECS besitzen ID und Version (insg. 48 bit).
    Man könnte sich die Frage stellen, warum es nicht einfach eine eindeutige hochzählende 48-bit Zahl ist, aber das wird
    aus performance Gründen gemacht.
    Innerhalb des ECS werden sogenannte `Tightly Packed Arrays' verwendet, um schnelle Iteration von Entitäten zu ermöglichen
    Diese Arrays enthalten Daten der Entitäten und die Entity-ID (ohne Version) ist der direkte Index dieser Arrays.
    Dadurch, dass Entity-IDs wiederverwendet werden (mit erhöter Version) kann sichergestellt werden, dass diese Arrays immer
    `Tightly Packed' sind und keine Lücken entstehen.
    Ansonsten würde sehr viel Speicher verschwendet werden.
    Wenn wir Entitäten löschen und neu erzeugen, bekommt diese Entität ggf. also eine bereits verwendete ID allerdings
    mit einer neuen Version.
    Eine Lösung wäre sicherzustellen, dass keine IDs wiederverwendet werden, das geht mit einem Config-Flag.
    Das würde dazu führen, dass `nur' $2^{32}$ eindeutige Entity-IDs genutzt werden können.
    Außerdem währe es Speichertechnisch sehr ineffizient aus Gründen wie oben beschrieben, sollte man also nicht tun
    Das sind $4.294.967.296$ Entitäten, was knapp werden könnte, wenn IDs eindeutig sein sollen (auch über den Prozess hinweg).

    \paragraph{}
    Die Entity-ID zu verwenden erzeugt allerdings noch ein weiteres Problem.
    Nehmen wir an, wir haben eine Zeitlang gespielt und aktuell sind Entities mit Entity-ID 150--200 aktiv auf der Leiterplatte.
    Entitiets 0--150 wurden gelöscht.
    Wenn wir das Spiel jetzt speichern und wieder laden, werden Entities mit ID 150--200 ins ECS geladen.
    Flecs würde, ohne zu zögern, neuen Entities die IDs 0--149 zuweisen, die sind ja noch nicht benutzt.
    Flecs weiß ja nichts davon, dass wir in unserer History diese Entities durchaus verwenden.
    Das kann gelöst werden, indem man in Flecs die Entity-ID range für neue Entities beim Start des Programms so konfiguriert,
    dass nur wirklich unbenutzte IDs verwendet werden (alle ab der bisher höchsten verwendeten ID).
    Das zweite bekannte C++ ECS Framework EnTT löst manche Probleme, indem man den Entity-Typen frei konfigurieren kann,
    z. B. als 64 Bit ID, von welcher 63 Bit für die ID und 1 Bit für die Version verwendet wird.
    In EnTT kann man allerdings keine ID-Range konfigurieren – es fängt immer von 0 an.

    \paragraph{}
    So oder so unterstützt Flecs also zu wenige Entitäten.
    Das Problem mit den ungenutzten Versionen besteht immer noch und somit werden Unmengen an Speicher verschwendet und
    die Iteration der Entitäten wird sehr, sehr langsam.
    Und so oder so habe ich nicht das Gefühl, dass die ECS Entity-IDs dafür gedacht sind über die Programmlaufzeit hinaus
    eindeutig und einzigartig zu sein.
    Die Entity-IDs sollten wie vom ECS gedacht verwendet werden, mit allen Speicher- und Zugriffsoptimierungen wie nur möglich.

    \paragraph{}
    Alles in allem wäre es also zu überlegen, die Entity-ID Entity-ID sein zu lassen und für die Entitäten, die History usw.
    eine ID - z. B. eine eigene globale ID ins Leben zu rufen.
    Diese könnte über eine Komponente den Entitäten zugewiesen sein.
    Man müsste dann allerdings zusätzlich ein Mapping globale ID -> Entity-ID verwalten um von einer Globalen ID auf
    die aktuelle Entity-ID zu kommen, was nicht weiter schwierig ist.
    Das bedeutet allerdings auch, dass wir in ECS Komponenten keine anderen ECS-Entity IDs einfach so mehr speichern können,
    sondern immer den Umweg über die globale ID gehen müssen, die Entity-ID könnte sich ja schließlich ändern!
    Das verhindert uns die Nutzung von manchen Flecs' Features, die eigentlich sehr angenehm wären (Relationships, welche Entities in Verbindung setzen).
    EnTT zum Beispiel stellt allerdings gar kein Relationship Feature bereit, da müsste man solche Dinge also eh selbst machen.

    \paragraph{}
    Durch die Verwendung von Arrays in ECS, die von Entity-IDs indiziert sind, ist es also faktisch eigentlich nicht möglich,
    einzigartige, automatisch-wachsende IDs als Entity-IDs zu verwenden, egal in welcher ECS implementation.
    Wenn man also die Vorteile des ECS nutzen möchte, ist eine zusätzliche ID zwingend notwendig.
    Das hätte zum Beispiel auch den Vorteil, dass die Entity-IDs auf den Clients nicht unbedingt mit den Entity-IDs auf dem Server übereinstimmen
    müssen, da alles über die globale ID geregelt wird.
    Das würde in Szenarien wichtig werden, in welchen der Client zusätzliche, nur lokal eine Rolle Spielende, Informationen im ECS
    mit abspeichern möchte.

    \subsection{Die Verbindung zwischen Grafik und Simulation}
    Die Simulation im ECS zu verwalten macht keinen Sinn.
    Hinter einer grafischen Node stecken ggf. hunderte Simulations-Nodes.
    Diese Änderungen im ECS zu speichern würde viel zu viel Speicher brauchen.
    Der Simulationszustand soll so oder so nicht in der History gespeichert werden.
    Nur bei Copy/Paste soll ggf. der Simulationszustand der kopierten Komponenten beibehalten werden.
    Um mit der ECS History zusammenzuspielen, sollten die Simulations-Nodes allerdings wirklich in den einzelnen
    ECS Entities gekapselt sein.
    Es sollten keine Simulations-Nodes zwischen den ECS-Entitäten referenziert werden.
    Die Pointer zu Simulations-Nodes könnten sich ja bei Löschen und Wiederherstellen einer Node ändern, da dann
    neue Simulations-Nodes erzeugt werden.
    Wenn eine Verbindung zwischen Komponenten getrennt wird, muss sichergestellt werden, dass die Simulations-Nodes auch richtig getrennt
    werden.
    Jedes ECS-Entity könnte Pin Entitäten bekommen, welche Kinder des Komponenten-Entities sind und die Pins der Simulation referenzieren.
    Dann kann das ECS einfach Verbindungen durch die Netzwerk-Entitäten herstellen.
    Ein Observer beobachtet dann einfach das Erstellen bzw Löschen der Beziehungen und passt die Simulation an.
    

\end{document}
